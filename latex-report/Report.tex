\documentclass{book}%[openany]

\input{Preamble}% no \include here possible!

\begin{document}

\begin{titlepage}
	\centering
	\vspace{3cm}
	{\scshape\Huge\ Cryptology \par}
	\vspace{3cm}
	{\scshape\Large Assignment 1: Vigenere\par}

	\vfill
	{\Large\itshape JH Darbyshire and Bengu Erenler}
\end{titlepage}

\section*{Part B}

The following are the results of our decryptions:
% INSERT RESULTS HERE %

\section*{Part C}
\subsection*{Extended Alphabets}
We believe extending the number of characters from 26 to 32 in general makes a cipher more difficult to break, particularly from a brute force point of view. Similarly would we expect the level of difficulty to diminish if the number of characters was reduced.
\par Indirectly, extending the number of characters may make, however, a cipher easier to break if those additions brought about strong statistical elements. For example our methods rely on letter frequencies being a signature. If an alphabet of 26 characters existed with equal probability of character appearance in the plaintext then this would likely be more difficult to break than a 32 character alphabet statistically favouring some characters more than others.
\par A mathematical argument for this can be made in terms of entropy: suppose a random character (\(X_2\)) from a two character alphabet has equal letter frequency then the entropy is:
\[ H(X_2) = -\sum P(x_i)log_2(P(x_i)) = 1 \quad \text{where} \quad P(x_i)=0.5 \quad \forall \; i \;.\]
The entropy for a similar character \(X_3\)in a similarly distributed set of three letters is:
\[ H(X_3) = 1.58 \;.\]
Whereas the entropy for a character \(X_{3*}\) three character set with probability density \{0.1, 0.1, 0.8\} is:
\[ H(X_{3*}) = 0.92 \;. \]
In this final example we would consider the unpredictability of any given character less than in the original case with two characters.

\subsection*{Cipher Adjustments}
\subsubsection*{(a)}
This is not sensible
\end{document}